% Discussion Chapter

The structure of a text is based on a pre-designed content structure (table of contents and the body) that can vary a great deal from thesis to thesis depending on the topic as well as the scope of the thesis. The presentation order of the first few pages is fixed, and should be presented as it is described in Sections\DIFdelbegin \DIFdel{\ref{sec:first_pages} to }\DIFdelend \DIFaddbegin \DIFadd{~\ref{sec:first_pages} to~}\DIFaddend \ref{sec:appendices} in these instructions. When applying page numbering, the title page has page number ‘1’, but page numbers are shown (at the right upper corner of each page, Arabic numerals) only from \charef{introduction}, Introduction, onwards.

\section{First pages}
\label{sec:first_pages}
The first pages include the title page, abstract(s), table of contents, foreword and list of abbreviations and symbols. 

\subsection{Title page}

In case the title page template of these instructions can't be used, you can produce the title page yourself following the model available from the degree program web pages. Please ensure that there are no typos, and that your degree program is stated correctly. Your name should be written into the middle. The title should be written below your name with capital letters, centered, and divided into several lines as necessary to produce a balanced appearance. University logo and the faculty name are placed at the top and the type of the thesis, degree programme name, and date are placed at the bottom.

\subsection{Abstract}

The abstract of your thesis will be fed into various databases and catalogues. It should crystallize the essence of your thesis. A good abstract is a bait that attracts the reader to take a closer look of the content. The abstract should be self-contained, i.e., the reader should be able to get a clear picture of your thesis from the abstract alone. There must not be any references to your thesis or other sources, but it should also not include any information not found in your thesis. The abstract should include the main elements, as well as the methods used and results obtained, and main conclusions of your thesis. The recommended length of an abstract is 200 words. Rare terminology and abbreviations should be avoided. 

The bibliographic information of the thesis should be printed at the top of the abstract page. The keywords of your thesis should be printed below the abstract. The recommended number of keywords is \DIFdelbegin \DIFdel{2-6 }\DIFdelend \DIFaddbegin \DIFadd{2--6 }\DIFaddend keywords or word sets. It is recommended that the keywords are not any words included in the title of the thesis. Keywords serve an important purpose for anyone performing literature searches in a library and other information catalogues. An example abstract is presented in Appendix 1.

\subsection{Abstract in finnish}

You also have to write your thesis abstract in Finnish, or have your abstract translated into Finnish. It should be written in flawless Finnish. If the language of the thesis is English, the abstract in English is placed first, and abstract in Finnish (Tiivistelmä) after it. If the language of the thesis is Finnish the order is opposite. The requirement of the Finnish abstract does not apply to international master’s degree students.

\subsection{Table of contents}

The table of contents lists the chapters with their headings and subheadings and their respective page numbers. However, the page number is shown in the table of contents only from the first chapter, Introduction, onwards.

\subsection{Foreword}

The foreword page should describe the aim of the thesis, and its various research stages, and present the partners, funding and circumstances involved in the thesis project. The forewords should also include words of gratitude, addressed to people who have been incremental in your thesis-writing process. The supervisor, the second examiner, and the technical supervisor can be mentioned as well.

\subsection{List of abbreviations and symbols}

\DIFdelbegin \DIFdel{All abbreviations and symbols used in the thesis have to be listed on this page. You should first explain all mathematical (and other similar) symbols , then abbreviations, so that Latin, Greek, etc. letters are all grouped separately. Abbreviations need to be explained in
text as well, when they are used the first time}\DIFdelend \DIFaddbegin \DIFadd{The abbreviations of terms, variables and measurement units as well as
the definitions of other symbols are given as a single list in
alphabetical order stylized exactly as in the body text (see
Section~\ref{sec:core_text}) on this page}\DIFaddend . You should check the validity
of all abbreviations and symbols from \DIFaddbegin \DIFadd{standards and other }\DIFaddend reliable
sources. Concerning measurement units, you should apply the
internationally approved SI-system of symbols\DIFaddbegin \DIFadd{~}\DIFaddend \cite{siopas,
systemofunits} and quantities and units specified in the IEC 80000-13
standard. For example, 1000 bytes is one kilobyte (1 kB, \DIFdelbegin \DIFdel{'}\DIFdelend \DIFaddbegin \DIFadd{`}\DIFaddend B' is Bytes,
\DIFdelbegin \DIFdel{'}\DIFdelend \DIFaddbegin \DIFadd{`}\DIFaddend b' is bits) and 1024 bytes is one kibibyte, 1 KiB. Similarly, 1 048
576 bytes mebi (Mi); the larger units are gibi (Gi), tebi (Ti), pebi
(Pi), and exbi (Ei).

\DIFaddbegin \DIFadd{In the list, the abbreviations are defined first, then mathematical and other symbols and last letter symbols so that Latin, Greek and other letters are each presented in their designated groups. 
}

\DIFaddend \section{Introduction}

In the introduction, you should describe the background and motivation of your thesis, introduce the reader to your research questions and methodology, describe in detail the objectives of your thesis (i.e. what is your aim to achieve), and on what basis the scope of your research area has been chosen. You can also state explicitly what will not be covered. You should cite earlier work done in the field. You should \textbf{not} discuss the results of your thesis in the introduction.

You can start by introducing some need of an individual user or the society, or some well-known application or technology. Specifying your own topic as part of such larger entity helps the reader to understand the work and estimate its significance and success. Note that you should avoid specifying too challenging objectives, as failing to meet the objectives can reduce the grade.

Nowadays master’s theses in engineering are often a part of wider research projects at universities or industries, and it might therefore be difficult for the reader to discern when the author is describing her/his personal work, and when is she/he describing the work of a research group. In cases like these, the author should describe as well as possible what exactly was her/his role and contribution in the research/project. At the end of the introduction, you might want to give an overview about the structure of your thesis.

\section{Core text part of your thesis}
\DIFaddbegin \label{sec:core_text}
\DIFaddend 

How you handle the core topic of your thesis depends essentially on the nature of your research/project. Most theses first describe the technological and scientific environment of the thesis and state-of-the-art related to the topic. This part should focus on presenting context for the author’s own work and for the decisions made by the author, so that it is later clear why one approach, technology, or method was chosen over another.

When the author develops a part for a larger system, it is necessary to describe this larger system as well. For example, when an author develops a software component for an LTE base station, the base station and LTE telecommunication network need to be described. However, the author needs to pay attention to describe the larger system only at a reasonable level of detail. The rule of thumb is that detailed information should be given only when this information is necessary to understand the rest of the thesis.

In the beginning, you can describe and analyze optional approaches or methods, e.g., by applying system-level modeling. As stated above, you can also present solutions proposed in literature sources. The best grades require presenting the state of the art of both technology and science---that is, presenting both the already available solutions and the ongoing research. Such presentation shows that the author possesses good knowledge on the thesis’ topic. Hence, it is crucial to get familiar with the latest literature on the thesis’s topic as soon as possible. Books, scientific articles, and patents can be searched from the sources offered by the university library. The library’s subject guide on information technology\footnote{ http://libguides.oulu.fi/c.php?g=58694} is a good starting point for searching literature.

\DIFdelbegin \DIFdel{Some thesis can include a theoretical analysis of a given topic , in which you present the grounds on which you analyze the topic, based either on literature sources or your own reasoning. The theory you present should be closely }\DIFdelend \DIFaddbegin \DIFadd{A theoretical approach on the given topic can become a natural element in some theses where the student presents, based on his on her own prior knowledge or referring to literature, the theoretical grounds on which the
thesis work relies. However, unnecessary wordiness must be avoided and as a result the theory presented must be directly }\DIFaddend linked to the \DIFdelbegin \DIFdel{focus of your thesis. }\DIFdelend \DIFaddbegin \DIFadd{realised work. It is noteworthy, that many theses do not include a theoretical section of any type. }\textbf{\DIFadd{Hence, you should not insist on adding theory if it does not fit in naturally.}}
\DIFaddend 

\DIFdelbegin \DIFdel{Mathematical equations and
denotations often play an important role in describing a theory}\DIFdelend \DIFaddbegin \DIFadd{In the possible presentation of related theory, equations and
mathematical notations often play a prominent role}\DIFaddend . However, \DIFdelbegin \DIFdel{one must bear }\DIFdelend \DIFaddbegin \DIFadd{it is
important to keep }\DIFaddend in mind that mathematics is a \DIFdelbegin \DIFdel{useful tool in writing , and not an end purposein itself. It is not necessary to present theory and equations in detail when they can be found from the literature---proper referencing suffices in such situations}\DIFdelend \DIFaddbegin \DIFadd{tool for technical
writing – not the main purpose}\DIFaddend . \DIFaddbegin \DIFadd{Therefore, introducing all details
with a mathematical equation is unnecessary. }\DIFaddend It is \DIFdelbegin \DIFdel{more crucial to
present general principles about the methods and solutions that are essential for the thesis. Usually it is not necessary to prove theorems or to derive the methods in detail but it suffices to present the }\DIFdelend \DIFaddbegin \DIFadd{sufficient to
present the }\DIFaddend basic equations, \DIFdelbegin \DIFdel{the relevant variables , }\DIFdelend \DIFaddbegin \DIFadd{necessary variables }\DIFaddend and the end
results. \DIFdelbegin \DIFdel{However, when the proving or derivation is done by the author her/himself, details need to be described. You can also include lengthy derivations as appendices if you need to .
}\DIFdelend \DIFaddbegin \DIFadd{If needed, the full derivation can be added to the thesis as
an appendix. In science and technology, two types of equations are
employed:
}\DIFaddend 

\DIFdelbegin \DIFdel{When presenting numerical results, the unit should }\DIFdelend \DIFaddbegin \begin{itemize}
\item{equations between quantities where letter symbols represent physical quantities, and}
\item{equations between numerical values where letter symbols represent the numerical values of the variables.}
\end{itemize}

\DIFadd{A quantity is the product of the numerical value and the unit of
measure. The unit of measure must }\DIFaddend always be separated \DIFdelbegin \DIFdel{with a space from the
numerical value preceding it }\DIFdelend \DIFaddbegin \DIFadd{from the
preceding numerical value with a spac }\DIFaddend (e.g., 5 Gbit/s\DIFdelbegin \DIFdel{). The mathematic }\DIFdelend \DIFaddbegin \DIFadd{,  5\,$^\circ$C, yet 5$^\circ$). Equations between quantities are recommended because they are not dependent on the choice of units of measure, unlike equations between numerical values. Calculations with quantities follow the rules of algebra, and the utilized symbols are typically single letters. The mathematical }\DIFaddend variables and symbols \DIFdelbegin \DIFdel{used in equations should be }\textit{\DIFdel{italicized}}%DIFAUXCMD
\DIFdel{. Vectors and matrices should be }\textbf{\textit{\DIFdel{italicized}}%DIFAUXCMD
} %DIFAUXCMD
\DIFdel{in bold.
Numbers}\DIFdelend \DIFaddbegin \DIFadd{of quantities must be italicized, i.e., written in italics (e.g., area $A$, electric charge $q$). Vectors must be italicized and boldfaced (e.g., acceleration $\mathbf{a}$).
}

\DIFadd{Numbers, functions, chemical symbols, units of measure, and subscripts are not italicized (e.g., 200}\DIFaddend , \DIFdelbegin \DIFdel{units}\DIFdelend \DIFaddbegin \DIFadd{$\cos$}\DIFaddend , \DIFdelbegin \DIFdel{and sub-indices or subscripts should not be italicized (sub-indices or subscripts that include symbols of equations should all be italicized ). Greek letters should not be italicized . Every equation should be a part of }\DIFdelend \DIFaddbegin \DIFadd{$\log$, NaCl, $\mu$A, cm, electron mass $m_\mathrm{e}$), but subscripts representing symbols of quantities are italicized (e.g., thermal velocity $v_\mathit{T}$). The lowercase Greek letters used as symbols of quantities are italicized, but uppercase letters are not italicized (e.g., $\delta$ vs.\ $\Delta$). More instructions for writing symbols can be found in the SI Guide~\mbox{%DIFAUXCMD
\cite{SIguide}}\hskip0pt%DIFAUXCMD
, and some of them may be discipline-specific.
List the symbols in the list at the beginning of your thesis, stylized the same as they appear in text and equations. Each equation must be part of a }\DIFaddend complete sentence. \DIFdelbegin \DIFdel{An empty row should be left at the top and at the bottom of }\DIFdelend \DIFaddbegin \DIFadd{A blank line is left above and below }\DIFaddend an equation, and \DIFdelbegin \DIFdel{equations should be numbered by applying running numbering, from the beginning to the end of the thesis. However, if you have a large number of equations, you can number them per chapter. The numbering should take place at }\DIFdelend \DIFaddbegin \DIFadd{the equations are numbered in an increasing order through the entire text or, alternatively, by chapter if there are large quantities of equations. The corresponding order number is placed in parentheses on }\DIFaddend the right side of the equation\DIFdelbegin \DIFdel{in parentheses. You should refer to the equation in the textby referring to the }\DIFdelend \DIFaddbegin \DIFadd{. In the text, the equations are referred to by the order }\DIFaddend number of the equation \DIFdelbegin \DIFdel{, e.g.: "As seen in \eqref{eq:speed}, the.
.."
}%DIFDELCMD < 

%DIFDELCMD < %%%
\DIFdel{An example: }\DIFdelend \DIFaddbegin \DIFadd{in parentheses.
}\DIFaddend In steady movement, speed $v$ is
\DIFdelbegin %DIFDELCMD < 

%DIFDELCMD < %%%
\DIFdelend \begin{equation}
    v = \frac{s}{t},
    \label{eq:speed}
\end{equation}
\DIFdelbegin %DIFDELCMD < 

%DIFDELCMD < %%%
\DIFdel{where $t$ is the amount of time required by the movement, and }\DIFdelend \DIFaddbegin \DIFadd{where }\DIFaddend $s$ is the distance \DIFdelbegin \DIFdel{.
}\DIFdelend \DIFaddbegin \DIFadd{and $t$ the time travelled. The implications of equation~\eqref{eq:speed}, in turn, are\ldots
}\DIFaddend 

After the opening sections of your thesis, you should present the actual personal contribution your thesis is making, although some of this can already be included in the previous chapters. Describing the phases of the development work provides a clear structure, that is, describing first the requirements for the solution, then the design and implementation, followed by testing (i.e. verification).

The requirements can be either given by the customer or the author might need to determine the requirements her/himself by interviewing experts, for example. In either case, the actual work must not be straightforward following of instructions given by others but the author must provide a significant contribution to the solutions. Thesis work is independent, analytical problem solving and engineering work, in which the author weighs different solutions for the encountered problems and presents the reasoning and justifications for the selections she/he make during this process. Essential selections can be presented, for example, in a table listing the alternatives and selection criteria, and giving justified points for the alternatives for each criterion. For example, when a certain device needs to be selected, the criteria can be price, performance, and reliability, and each candidate devices is given points for each criterion.

The design should be presented using the established tools and representations. For example, UML diagrams can be used to describe software design. The aim of testing is to verify the solution. When the requirements are presented in an earlier chapter, it is natural to test the solution against the requirements. So that your observations will benefit your readers, you should record and publish them in your thesis in as thorough and detailed a way as possible in their original format (such as tables, etc.). You must be careful not to mix actually obtained results and your estimates of the results. The reliability of the results can be analyzed using statistical methods, though strong conclusions require large samples.

In construction, or software-based theses, you should approach your research questions by means of system planning. You should only describe the necessary details of your basic theory and construction, which are directly relevant to the construction or software. You can include a broader description of the constructions and/or software in an appendix if you deem it important. You should describe the details of your construction/software in sections, starting from a general level, and then going into more detail, level by level. You should avoid a too detailed description concerning the operational principles of software structures and electronic circuits. However, details that are essential to the thesis---especially details that cover new ground or are not clear even to experts in the field---should be included in the thesis.
\DIFdelbegin %DIFDELCMD < 

%DIFDELCMD < %%%
\DIFdel{Measurements are in most cases essential to verify the developed solution and they should be well planned. A master’s }\DIFdelend \DIFaddbegin \DIFadd{Measurements are intrinsically related to theses which present constructions and therefore they must be carefully planned. The same is valid for the testing of software solutions. Yet, the }\DIFaddend thesis is not a measurement report \DIFdelbegin \DIFdel{, though, and all results are not necessary to report. Instead, every figure and every table you present should }\DIFdelend \DIFaddbegin \DIFadd{and hence it is unnecessary to introduce all the results. Consequently, each presented figure must }\DIFaddend have a clear \DIFdelbegin \DIFdel{statement that you want to show}\DIFdelend \DIFaddbegin \DIFadd{significance}\DIFaddend .

These instructions give guidelines on writing the core text of your thesis when your topic is to develop a concrete solution that can be tested, like a device, piece of software, or an algorithm. Even if the work does not produce such a concrete result, the analytical problem solving has to be present in the documentation and the result be presented clearly. For example, for a literature study, presenting the subject matter and the essential content found from the literature is not enough, but the author has to classify this content or otherwise analyze it. In topics not producing concrete results, such analysis is a mandatory requirement for the best grades. Specifically, it is always crucial to analyze how well the results meet the objectives set in the introduction. This analysis can be placed in the discussion chapter.

\section{Discussion}

A good thesis or other scientific work always has a discussion. In order to write this, you should be able to look at your work as if from a distance, to \DIFdelbegin \DIFdel{"step out of the box"}\DIFdelend \DIFaddbegin \enquote{step out of the box}\DIFaddend , as they say. You should be self-critical, compare your work to similar work in the field, and think analytically. You should be able to crystallize the results of your work, and put them into words. This can often be difficult even for an experienced writer, but it helps if you are well acquainted with published literature in the field.

A justified analysis of meeting the thesis’ objectives is an important part of the discussion, and generally analysis of the results. When the requirements for the developed solution are presented earlier in the thesis, it is natural to first discuss how these requirement were fulfilled and from that conclude how well the objectives were met. Own solution should also be compared to the state of the art (technology and science) presented at the beginning. Your solution does not have to be better, but this kind of analysis is essential in engineering work and hence an important part of master thesis as well. Such a comparison is needed for the best grades.

The analysis can contain also more general commenting, for example, what was easy and what was difficult in the work. You can also discuss the overall significance of your thesis on a more general level. Outlining potential further development based on your thesis, is valuable, especially if you have put forth clearly new or groundbreaking ideas. However, you should avoid unnecessary speculation here, as well as elsewhere: all statements should be well reasoned and brief. Other requirements for discussion can be found from the supervisors’ evaluation instructions that are available at the degree programme web pages\DIFaddbegin \DIFadd{~}\DIFaddend \cite{mscstudies}.

\section{Summary}

In the summary (or conclusions) you should present clearly in a nutshell the objectives of your thesis, its main content, your results, and the significance of your results. You should lay special emphasis on your results if you feel you have accomplished something. In summary you should not make references, or present any results not found elsewhere in your thesis.

The abstract and the summary overlap to a certain extent; they both describe the main contents and results of a thesis. However, the nature of the summary is broader and it does not have to be self-contained. In it, you should describe your aims, and you could describe any optional solutions or approaches, and motive the choices you have made. The abstract on the other hand could just describe in detail the solutions and approaches chosen for the thesis, leaving the optional approaches out.

\section{References}

The use of references serves many purposes. Scientific method relies on familiarizing with the topic and state of the art. However, efficient referencing can also compress your text, as you can leave the details in the references and repeat only the most important results.

The literature survey should be close to exhaustive, and this means that most of the information you present is taken from references. If a piece of information is not derived or devised by you, it is borrowed, and the origin of the information must be stated. Presenting somebody else’s finding as your own is a scientific theft (plagiarism) that has serious consequences.

You should refer to original sources of the data---for example, to a book and not the handouts made based on the book. Be careful when referencing: the things you state really need to be found from the reference.

The references are mostly cited in your own words, and direct quoting is used only if you want to emphasis the source. In this case you place the quote in \DIFdelbegin \DIFdel{hyphens}\DIFdelend \DIFaddbegin \DIFadd{quotation marks}\DIFaddend , for example saying that the original phrasing of Moore’s law is  \DIFdelbegin \DIFdel{“}\DIFdelend \DIFaddbegin \DIFadd{``}\DIFaddend The complexity for minimum component costs has increased at a rate of roughly a factor of two per year (see graph on next page).\DIFaddbegin \DIFadd{''~\mbox{%DIFAUXCMD
\cite{moore}}\hskip0pt%DIFAUXCMD
. Longer
quotations (several lines) are placed in an indented paragraph (omit quotation marks). As Moore stated:
}\begin{quote}
\DIFaddend Certainly over the short term this rate can be expected to continue, if not to increase. Over the longer term, the rate of increase is a bit more uncertain, although there is no reason to believe it will not remain nearly constant for at least 10 years.\DIFdelbegin \DIFdel{” \mbox{%DIFAUXCMD
\cite{moore}
}\hskip0pt%DIFAUXCMD
.  }\DIFdelend \DIFaddbegin \DIFadd{~\mbox{%DIFAUXCMD
\cite{moore}
}\hskip0pt%DIFAUXCMD
}\end{quote}
\DIFaddend Correct referencing is important as the archiving of electronic \DIFaddbegin \DIFadd{theses }\DIFaddend makes them easily accessible (either in \DIFaddbegin \DIFadd{the }\DIFaddend library or in the Internet, depending on the publicity level selected by the thesis author). \DIFdelbegin %DIFDELCMD < 

%DIFDELCMD < %%%
\DIFdel{The list of references is written according to }\DIFdelend \DIFaddbegin \DIFadd{The reader must be able
to easily find the original source to both assess your work and for }\DIFaddend the
\DIFdelbegin \DIFdel{instructions of the IEEE Transactions series \mbox{%DIFAUXCMD
\cite{ieeetransactions}}\hskip0pt%DIFAUXCMD
; using the running numbering. Referencing is easiest to do using }%DIFDELCMD < \LaTeX%%%
\DIFdel{\ for writing the
thesis and following the model given in these instructions. You should add your own references to the file }\textit{\DIFdel{citations.bib}}%DIFAUXCMD
\DIFdel{.
Following the instructions automatically creates the list of references in
}\DIFdelend \DIFaddbegin \DIFadd{authors of the reference to be properly credited.
}

\DIFadd{The sources are cited and the bibliography is compiled according to
the IEEE style~\mbox{%DIFAUXCMD
\cite{ieeetransactions}}\hskip0pt%DIFAUXCMD
.  Sources are numbered in
increasing order and listed in }\DIFaddend the order in which they appear in the
text.  References in the text are then indicated with a reference
number, e.g.,\DIFdelbegin \DIFdel{\mbox{%DIFAUXCMD
\cite{lappalainen} }\hskip0pt%DIFAUXCMD
or }\DIFdelend \DIFaddbegin \DIFadd{~\mbox{%DIFAUXCMD
\cite{lappalainen} }\hskip0pt%DIFAUXCMD
or~}\DIFaddend \cite{lappalainen, acta, korpela}\DIFaddbegin \DIFadd{,
which is placed inside square brackets, inside sentence punctuation
with a (non-breaking) space before the bracket}\DIFaddend . References to books
should \DIFdelbegin \DIFdel{include also }\DIFdelend \DIFaddbegin \DIFadd{also include the }\DIFaddend page number, for
example\DIFdelbegin \DIFdel{\mbox{%DIFAUXCMD
\cite[p.15]{lappalainen} }\hskip0pt%DIFAUXCMD
or\mbox{%DIFAUXCMD
\cite[p.15-17]{lappalainen}}\hskip0pt%DIFAUXCMD
.
}\DIFdelend \DIFaddbegin \DIFadd{~\mbox{%DIFAUXCMD
\cite[p.~15]{lappalainen} }\hskip0pt%DIFAUXCMD
or~\mbox{%DIFAUXCMD
\cite[pp.~15--17]{lappalainen}}\hskip0pt%DIFAUXCMD
.
Figures and equations can be cited using:~\mbox{%DIFAUXCMD
\cite[Fig.~3]{lappalainen}}\hskip0pt%DIFAUXCMD
;
\mbox{%DIFAUXCMD
\cite[eq.~(3)]{lappalainen}}\hskip0pt%DIFAUXCMD
. When writing in Finnish, you should adapt the IEEE guidelines, e.g., \mbox{%DIFAUXCMD
\cite[s.~15]{lappalainen}}\hskip0pt%DIFAUXCMD
;\mbox{%DIFAUXCMD
\cite[s.~15--17]{lappalainen}}\hskip0pt%DIFAUXCMD
;
\mbox{%DIFAUXCMD
\cite[Kuva~3]{lappalainen}}\hskip0pt%DIFAUXCMD
;~\mbox{%DIFAUXCMD
\cite[Yhtälö~(3)]{lappalainen}}\hskip0pt%DIFAUXCMD
.
}\DIFaddend 

\DIFaddbegin \DIFadd{The reference is placed immediately where it is used or at the end of
the sentence it relates to. Grammatically it can treated as a noun or
a footnote, e.g., ``As shown by Lappalainen~\mbox{%DIFAUXCMD
\cite{lappalainen}}\hskip0pt%DIFAUXCMD
'' or
``As shown in~\mbox{%DIFAUXCMD
\cite{lappalainen}}\hskip0pt%DIFAUXCMD
''. The authors are only named if they
are relevant for the sentence. }\textbf{\DIFadd{If no citation is mentioned,
the sentence is assumed to be your own idea}}\DIFadd{.
}

\DIFadd{In IEEE style, if multiple sentences refer to the same source, the
citation should normally be repeated for each sentence to avoid
ambiguity. If it is made explicitly clear that the same source
continues -- for example, by using linking phrases (“According to Smith
}[\DIFadd{5}]\DIFadd{\ldots He states\ldots”) — then repeating the citation for each
sentence is not strictly necessary.
}

\DIFadd{If an entire paragraph is paraphrased from one source, you may place a
single citation after the final sentence (as demonstrated here by
placing~\mbox{%DIFAUXCMD
\cite{lappalainen} }\hskip0pt%DIFAUXCMD
at the very end of this paragraph). This
approach should be used sparingly in scientific writing, but it can be
acceptable in a thesis literature review where a subsection is devoted
to a single source. In all cases, ensure that paraphrased material is
written entirely in your own words. Avoid covering long page ranges
from books or articles in one block -- instead, summarize in smaller,
focused chunks with exact citations in each
sentence.~\mbox{%DIFAUXCMD
\cite{lappalainen}
}\hskip0pt%DIFAUXCMD
}

\DIFadd{Referencing is easiest to do using }\LaTeX\DIFadd{\ for writing the thesis and
following the model given in these instructions. You should add your
own references to the file }\textit{\DIFadd{citations.bib}}\DIFadd{. A bibliography
manager, such as Zotero~\mbox{%DIFAUXCMD
\cite{zotero,uniouluzotero}}\hskip0pt%DIFAUXCMD
, is strongly
recommended, as it often automatically fills in the required fields
of the references. 
}

\DIFaddend References need to be presented so that it is clear what information
is from a source and what is created by the author. Also the sources
of equations, figures, and tables need to be given, when not created
by the author her/himself. In addition, permissions need to be asked
for tables and figures. Special attention needs to be paid to web
sources, as web content can have a short life cycle. Hence, the date
of downloading referenced web content should always be mentioned in
the reference.

\DIFdelbegin \DIFdel{Examples }\DIFdelend \DIFaddbegin \DIFadd{The examples }\DIFaddend below illustrate correct references. All authors should
be listed, the names of journals and series should be written
completely without abbreviations, and only the first letter of the
title and proper nouns in the title should have uppercase letters. The
DOI (Digital Object Identifier) of the publication should be included
in the reference, when possible. This identifier locates the
publication in an unambiguous fashion and hence facilitates the work
of a reader searching the reference. \DIFaddbegin \DIFadd{The }\DIFaddend DOI can be found from the
publications front page or from publishers’ databases (e.g. IEEE
Xplore, Springer Link, ACM Digital Library). \DIFaddbegin \DIFadd{When the reference has a
DOI, you do not have to include an URL for it.
}\DIFaddend 

The \textit{citations.bib} file in Overleaf ITEE MSc thesis template should guide you to use correct citation formats. \DIFaddbegin \DIFadd{The initials of the author's first names are written first, followed by the last name. }\DIFaddend If you are using other text preparation tools such as Microsoft Word, you should cite articles in journals  as\DIFaddbegin \DIFadd{~}\DIFaddend \cite{ojala:2002}, in series as\DIFaddbegin \DIFadd{~}\DIFaddend \cite{riekki:1998}, and in books as\DIFdelbegin \DIFdel{\mbox{%DIFAUXCMD
\cite{pietikainen:2011}}\hskip0pt%DIFAUXCMD
}\DIFdelend \DIFaddbegin \DIFadd{~\mbox{%DIFAUXCMD
\cite[p. 55]{pietikainen:2011}}\hskip0pt%DIFAUXCMD
}\DIFaddend . A section of compiled work should be cited as\DIFaddbegin \DIFadd{~}\DIFaddend \cite{cvejic:2005}, and articles in conference proceedings as\DIFaddbegin \DIFadd{~}\DIFaddend \cite{heikkila:1997}. The citation to a thesis should look like\DIFaddbegin \DIFadd{~}\DIFaddend \cite{heikkinen:2011}, and patents should be cited in the manner of\DIFaddbegin \DIFadd{~}\DIFaddend \cite{toivonen:2004}. In some cases it is necessary to cite a source in the Internet\DIFdelbegin \DIFdel{\mbox{%DIFAUXCMD
\cite{roning:2011}}\hskip0pt%DIFAUXCMD
}\DIFdelend \DIFaddbegin \DIFadd{~\mbox{%DIFAUXCMD
\cite{korpela}}\hskip0pt%DIFAUXCMD
}\DIFaddend , and publications that lack a known author\DIFdelbegin \DIFdel{\mbox{%DIFAUXCMD
\cite{asumistaso:1991}}\hskip0pt%DIFAUXCMD
}\DIFdelend \DIFaddbegin \DIFadd{~\mbox{%DIFAUXCMD
\cite{asuntoliitto_asumistaso_1969}}\hskip0pt%DIFAUXCMD
}\DIFaddend . Please, see the References \DIFdelbegin \DIFdel{Chapter for the models}\DIFdelend \DIFaddbegin \DIFadd{chapter for examples}\DIFaddend .

\section{Appendices}
\label{sec:appendices}
\DIFdelbegin %DIFDELCMD < 

%DIFDELCMD < %%%
\DIFdelend Things you can include as an appendix are, e.g., derivations of equations or formulas, details of important computer programs, various tables, or performance characteristics and descriptions of special equipment or components applied in the thesis work. You can also include construction drawings and parts of catalogues in the appendix. Appendices are titled as shown before. As with figures and tables, all appendices you include should have a clear meaning---the number of appendices itself is not a merit.

If your text seems to contain a lot of references to an appendix, it may be easier for the reader that you copy the information (e.g., a schematic) into the text. This way the reader does not need to browse between pages.

Large block diagrams or schematics can be copied on A3 size paper. This needs to have two bends (Z-like) so that the right side of the appendix lies top and with a width of ca 2/3 of A4 page.
