The process starts with specifying the topic of the thesis and applying for its formal approval by the degree programme committee head. You need to define the topic together with the commissioner and the supervisors (see the next section). The topic, commissioner, and supervisors are specified in the topic application form that is used to apply the approval of the topic. The list below presents the general thesis process, with emphasis on the administrative issues and supervision:

\begin{enumerate}
    \setlength\itemsep{0pt}
    \setlength\parskip{0pt}
    \item approval for the topic,
    \item writing the thesis (preferably at the same time as carrying out the research),
    \item delivering the thesis and applying for the degree,
    \item evaluation and grading, and
    \item granting the degree.
\end{enumerate}

The second stage is the most laborious part, which includes writing the thesis and performing the actual work, for example, designing, building and testing a piece of software. When the thesis is ready, it is reviewed by the supervisors, modified by you if necessary, and then delivered to the Laturi system. The next step is to apply for the degree certificate and to perform the other tasks that are required from graduating students. The supervisors download the thesis from the archive and evaluate it. The degree programme committee approves the grading in its meeting and the faculty grants the degree. The thesis is transferred to the university archive and published according to the publicity level you defined when uploading the thesis to Laturi. The graduation ceremony ends your M.Sc. studies. The process is explained in more detail in the rest of this document and on the degree programme web pages~\cite{mscstudies}.

\section{Approval of the topic}

Your first task is to find a commissioner for your thesis that is typically a company which hires you to make a thesis project. For example, a summer job or training in a company or some other organization gives a good starting point to continue in a thesis project. A research unit in the Faculty of Information Technology and Electrical Engineering can also act as the commissioner, in which case the topic will most likely be linked with ongoing research at the unit. Research units provide information about current master’s thesis topics in their web pages. You can also directly ask topics from the research unit personnel. In some cases the research units can pay salary to the master’s thesis worker, but in most cases you need to be prepared to make the thesis for free.

There are 2-3 supervisors involved in the master’s thesis process, and they are assigned after the topic is confirmed. \textit{Principal supervisor} is a professor or a  doctor who belongs to the personnel of some ITEE research unit. A list of potential supervisors and their fields of expertise can be found from the degree program web pages~\cite{mscstudies}. The professor or doctor you approach first can decide based on  the thesis topic to act as a principal supervisor her/himself or suggest someone else.  Usually, the thesis is done on a topic within the area of your orientation (major).  However, topics proposed by companies are often multi-disciplinary or cross-scientific  and the topic does not fit into the realm of any particular orientation. In these cases  the supervision should be agreed with a professor or doctor who best represents the  overall field of the thesis work.

If the master’s thesis is made to a university research unit, the commissioner often becomes the principal supervisor. If the thesis is made to a company or some other organization outside the university, the commissioner should assign a \textit{technical supervisor} who is responsible for advising you in the technical matters. It is also common to have a technical supervisor for theses made to university research units. In that case the role of the principal supervisor is mainly to ensure that the requirements set for a master’s thesis are fulfilled. In addition, there is always a \textit{second examiner} from the university who is selected by the principal supervisor. The second examiner evaluates the thesis together with the principal supervisor and sometimes also participates to supervision in the issues related to her/his expertise.

If the thesis work is conducted outside the university, it is important to find the principal supervisor as soon as possible in order to make sure that the topic is suitable and to agree on the scope of the thesis. It is recommended to organize a meeting together with the commissioner and the principal supervisor to discuss about the thesis work before it has been started. It is also recommended to write a short project plan that describes the background, motivation, and objectives of the work as well as a timetable and possible risks. The project plan gives a good basis for following the progress of the work.

After the supervisors have been selected, invite the main supervisor in Laturi. The head of the degree programme will will formally appoint the role. After the aims have been agreed with the supervisors, you are ready to submit the research plan for your thesis.  You should apply for approval as soon as possible, to be able to take the possible feedback into account in your thesis work. 

As a rule, Bachelor's thesis is written in Finnish. It can be written in English in exceptional cases, for example, if the supervisor is English-speaking or if the thesis project group includes members who don't speak Finnish. In English-language degree programmes the Bachelor's thesis is written in English. For Master's thesis, the student can choose the language. If the work is written in English, proofreading by a professional translator can be required. The decision is made by the principal supervisor. The author is responsible for paying the proofreading costs unless it is specified otherwise in the application. It should be also noticed that language is one of the thesis evaluation criteria.

\section{Writing the thesis}
\label{writing}

During the actual work, you are expected to communicate regularly with the technical and principal supervisors. In the beginning, meetings with the principal supervisor can be less frequent, but during the write-up phase, regular meetings are more important.

Every master’s thesis is a unique project and it is difficult to define general rules what the individual steps are and how frequently the meetings with the supervisors should be arranged. Master’s thesis gives an opportunity for demonstrating your professional maturity, but it is also the final learning possibility at the university. Hence, it is important to find the balance between the independent work and the amount of supervision required, which is one of the thesis evaluation criteria. You are entitled to get the necessary supervision, but do not expect that the supervisor will make the thesis for you. During the actual work it is recommended to have frequent appointments with the technical supervisor for discussing about the issues that help you to solve the problems encountered in the work. These meetings also keep the supervisor updated and ensure that you are not stuck or going to wrong direction.

At the beginning of the write-up phase, guidance from the principal supervisor is crucial concerning the structure, presentation order and style of the thesis. During the write-up, meetings with the principal supervisor should be held in order to discuss a) whether the information order and emphases are right, b) whether the issues being covered or planned to be covered are relevant to the thesis, and c) whether some areas have been overlooked. The emphasis in the meetings with the principal supervisor should mainly concern the structuring of the thesis.

At minimum, you should present the table of contents to the principal supervisor before you start writing the thesis, present the first complete draft when it is ready and in this meeting agree the necessary modifications, and then review the modifications in follow-up meetings. In these final meetings (one or more), special attention needs to be paid to correct referencing of others’ work and to the rights and permissions to the content included in the thesis. When the supervisor judges the thesis to be ready, she/he permits electronic publication of the thesis.

Further, always discuss the use of Artificial Intelligence-powered tools with your supervisor(s) well before writing the thesis. You must include a dedicated subsection at the end of the Inroduction chapter, titled \textit{Author's Contributions and the Role of Artificial Intelligence}. The statement must first clearly clarify what was the role of the thesis author in the included work and, second, how was AI used in preparing the thesis. Please familiarise yourself with the more specific guidelines in Section '\textit{Clarifying contributions and using AI}' in Chapter~\ref{implementation} of this guide.

\section{Delivering the thesis and applying for the degree}

Starting from January 2013, master’s thesis have been published only in an electronic form. When the thesis is ready and the supervisor has given the permission, the student delivers an electronic version of the thesis to the Laturi system\footnote{http://laturi.oulu.fi/. A good name for your thesis file is <year>-<month>-<familyname>-<firstname>-Thesis.pdf. When several theses are downloaded for review from Laturi, this naming convention helps to identify the files.}. Since the supervisors fetch your thesis for evaluation from Laturi, MSc students need to deliver the thesis no later than 10 days prior to a degree programme committee meeting approving your thesis. You can discuss details of the schedule with the supervisor. Committee meetings are held mainly once per month, and the meeting schedule can be found from the degree programme web pages (for BSc students~\cite{bscstudies} and MSc students~\cite{mscstudies}). After submission to Laturi, BSc students will send the final version of the thesis to the study office as well.

The delivered version must be the final version of the thesis manuscript\footnote{A second version of the thesis can be uploaded to the Laturi system only for a cogent reason and requires a permission from the university library (this permission has to be applied by the student).}. The thesis must have the official title page and the layout and general structure presented in this document. \textbf{Pay special attention to the title page, as you have to set your degree program correctly.} The new template can be downloaded from degree programme web pages~\cite{mscstudies}; old templates are no longer valid.

Once the thesis is accepted by the supervisors and graded by the degree programme committee, it is transferred to the university archive and published at the OuluREPO repository\footnote{https://oulurepo.oulu.fi/} or e-thesis workstations in the university library according to the publicity level you determined when you delivered the thesis.

The application for a diploma is submitted via Peppi's graduation service at student's desktop. In addition, there is a feedback questionnaire for BSc students. MSc students can download TEK Survey for Graduates from~\cite{mscgraduation}. The secretaries will prepare your papers, calculate your GPA, etc. 

Before graduation you also need to attend the \textbf{maturity test.} Passing the maturity test is required for all degree students and it is taken after the completion of the thesis. The students with Finnish/Swedish as the language of their elementary education will write the test in Finnish/Swedish. If the elementary education has been taken in some other language, the maturity test is taken in English. The maturity test is a written examination based on the thesis, where the candidate is asked to write an essay about the topic(s). The registration for the maturity test should be discussed with the supervisor as you are getting your thesis approved. Finally, you need to return all books, keys, equipment, machinery, and tools that belong to the university.

\section{Evaluation and grading}

The principal supervisor and the second examiner download the thesis from Laturi for \textbf{final evaluation and grading.} They fill out the evaluation form with the proposed grading in Laturi. If the thesis was commissioned by a company or an organization outside the university, the technical supervisor is expected to fill out a separate evaluation form and send it to the principal supervisor well in advance before the grading. The degree programme committee approves the thesis based on the evaluation by the principal supervisor and the second examiner. The forms and a document describing the evaluation criteria can be found from the degree programme web pages~\cite{mscstudies}.

A master’s thesis that has been approved will be graded according to the university regulations with the following 5-point scale: satisfactory (1), very satisfactory (2), good (3), very good (4), and excellent (5). The title of the thesis, the name of the Degree programme director, as well as the grade will be printed on the Master’s Degree Certificate.

The student has the right to see the grade proposed by the supervisor, as well as the evaluation form, \textbf{three} days prior to the committee meeting. In case the student feels mistreated, she/he has the opportunity to issue a written \textbf{appeal} to the board of examiners of the University of Oulu concerning the evaluation of his/her thesis no later than \textbf{fourteen days} after having received the information. This process evidently delays the graduation.

\section{Publicity}
\label{sec:publicity}

A master’s thesis is a public document. A thesis must therefore not reveal any business secrets or confidential information. The thesis can only be evaluated based on its written contents. For this reason, it is crucial to discuss thesis publicity with company representatives as early as possible, when the topic is defined. Should any major conflicting interests arise between the author and the customer concerning the publication of information, the author should turn to the principal supervisor for consultation.

The Ministry of Education has issued a set of written statements concerning the public nature of master’s theses to universities and colleges. According to the statements the thesis must not contain classified information, and once approved, the thesis should be public. The ownership as well as the publication and/or patent rights should be agreed upon together with the supervisor, author, and the possible commissioner.

Starting from January 2013, all master’s thesis written at the University of Oulu have been published electronically by the university library. The publicity level is decided by the author when the thesis is delivered to Laturi; whether the electronic version will be available to everyone from an open access repository or only at special workstations in university libraries.

\section{Thesis awards}

The Technological Society of Finland and the IEEE Finland section issue the Best Thesis of the Year Award, chosen from a list of candidates that is compiled by Finnish universities and colleges and is including almost all theses published annually in Finland.
