Writing a thesis takes on a major part of completing the master’s degree studies. The thesis assignment prepares for the independent engineering work. Hence, supervision plays a smaller role in thesis procedure than during previous studies. A typical master’s thesis represents a solution to a relatively extensive technical problem. Additional studies in the given field are often necessary; however, the aim of the thesis work is to make use of the knowledge and skills acquired during preceding studies. Furthermore, technical and scientific documentation skills will be strengthened. The thesis work can be a conducted as a part of a larger project, but the master’s thesis itself should be written individually.

The thesis work is undertaken in the final phase of the studies. It is recommended to begin the thesis work during the autumn term of the 2\textsuperscript{nd} study year in the master’s programme or the 5\textsuperscript{th} year after starting in the bachelor’s programme. However, timing is flexible and it is also possible to begin earlier depending on advancement in studies. As a general rule, it is time to get started, when there are 15 to 30 credit points left of the total coursework (in addition to the master’s thesis). Some fields of study require certain courses to be completed before the master’s thesis. Requirements should always be checked in advance with the supervisor. Information on the degree specific requirements is provided by the secretary of the degree program and the registered credit points can be viewed in Peppi.

\section{Thesis guide}
The aim of these instructions is to give detailed guidelines for composing and writing a master’s thesis. This document describes the thesis work process; starting from searching the topic to the formal approval of the completed thesis. The process has otherwise stayed the same for quite a while, but the final stages were changed in the beginning of 2013 due to electronic archiving for all master’s theses at the University of Oulu. Hence, the older versions of this document are no longer valid and should not be used. Moreover, this document offers practical instructions for the writing of the thesis: for the literary structure, the layout, and for the writing process as well. This document itself has been edited according to the instructed layout, though the literary structure is not similar to that of a master’s thesis.

Before starting your master’s thesis work, \textbf{please first read this document carefully.} Following these instructions closely is likely to produce a better result and a higher grade. When the supervisor does not have to point out about these instructions (specifically about the layout) she/he can focus on instructing on the content and how the best grades can be achieved. The degree programme web pages \cite{mscstudies} offer instructions for master thesis as well. When neither of these sources gives an answer to your questions, please ask your supervisor or from the study affairs office. And remember: Although writing the thesis can be difficult at times, every M.Sc. graduated from Faculty of ITEE has succeeded in this effort!

\section{Author's Contributions and the Role of Artificial Intelligence}

The structure of a thesis can vary. However, at the end of the introductory chapter, you must include a section titled \textit{Author's Contributions and the Role of Artificial Intelligence}. In this section, you should clarify your independent contributions to the thesis and explain how AI was used in its preparation, if applicable. More information can be found in Sections~\ref{writing} and~\ref{AI}.
